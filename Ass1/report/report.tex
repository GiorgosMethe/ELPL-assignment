\documentclass[a4paper,11pt]{article}

\usepackage{fullpage}
\usepackage{color}
\usepackage{hyperref}
\usepackage{amsmath}
\usepackage{amssymb}
\usepackage{tikz}
\usepackage{tabularx}
\usepackage{booktabs}
\usepackage{amsmath}
\usepackage{multirow}
\usepackage{layouts}
\usepackage{array}
\usepackage{pgf}
\usepackage{tikz}
\usepackage{amssymb}
\usepackage{graphics}
\usepackage{eucal}
\usepackage{ifthen}
\usepackage{ifpdf}
\usepackage{lmodern}
\usepackage{amsthm}
\usepackage{epstopdf}
\usetikzlibrary{positioning}

\hypersetup{
  colorlinks,%
    citecolor=blue,%
    filecolor=blue,%
    linkcolor=blue,%
    urlcolor=mygreylink     % can put red here to visualize the links
}

\definecolor{hlcolor}{rgb}{1, 0, 0}
\definecolor{mygrey}{gray}{.85}
\definecolor{mygreylink}{gray}{.30}
\textheight=8.6in
\raggedbottom
\addtolength{\oddsidemargin}{-0.375in}
\addtolength{\evensidemargin}{0.375in}
\addtolength{\textwidth}{0.5in}
\addtolength{\topmargin}{-.375in}
\addtolength{\textheight}{0.75in}

\newcommand{\resheading}[1]{{\large \colorbox{mygrey}{\begin{minipage}{\textwidth}{\textbf{#1 \vphantom{p\^{E}}}}\end{minipage}}}}

\newcommand{\mywebheader}{
  \begin{tabular}{@{}p{5in}p{4in}}
  {\resheading{Project 1: Statistical Parsing, Steps 1 \& 2}} & {\Large 25 November, 2012}\\\vspace{0.2cm}
  \end{tabular}}

\begin{document}


\begin{center}
{\LARGE \textbf{Elements of Language Processing and Learning}}\\ [1em]
\end{center}
\mywebheader

\begin{center}
{\Large By:} \\ \vspace{0.1cm}
{\Large Georgios Methenitis, Stud.Id:10407537} \\ \vspace{0.1cm}
{\Large Marios Tzakris, Stud.Id:10407537}
\end{center}


\section{Introduction}
In this project we had to implement a prototype statistical parser. Having the given treebanks we had to parse and extract the rules using this statistical parser (Step 1), resulting to a probabilistic context free grammar (PCFg). Then, given this extracted grammar, we implement the Cocke-Younger-Kasami (CYK) algorithm, in order to parse some test sentences generating a parse-forest for each one of them. In Section~\ref{parser}, we are going to describe the implementation of our parser. In Section~\ref{cyk}, we  
are going to describe the implementation of the CYK algorithm as well as, our assumptions in respect to the unknown words, and a brief explanation of every step of this algorithm through pseudo-code. Section~\ref{concl}, serves as an epilogue to the first two steps of this project.
 


\section{Parser}
\label{parser}
Each rule is represented by the non-terminal left-hand side node, and the right-side node or nodes. a typical example of  binary rule is this: $X \rightarrow \alpha\	 \beta$. In general in our treebanks there were mostly binary rules but in some cases we had to handle and unary rules which have this form: $X \rightarrow \alpha$.




\section{CYK Algorithm}
\label{cyk}
agargbarebeatb

\section{Conclusion}
\label{concl}
In this first two steps, we have seen....

\end{document}